\documentclass{jsarticle}
\usepackage{amsmath, amssymb} % 数学記号のパッケージ
\usepackage{ascmac} % 囲み枠のパッケージ

% ハイパーリンクに関するパッケージとその設定
\usepackage[dvipdfmx]{hyperref}
\usepackage{pxjahyper}
\hypersetup{
 colorlinks=true
}

% 定理などの番号付けに関するパッケージとその設定
\usepackage{amsthm}
\theoremstyle{definition}
\newtheorem{defi}{定義}[section]
\newtheorem{prop}[defi]{命題}
\newtheorem{thm}[defi]{定理}
\newtheorem{supp}[defi]{補足}
\newtheorem{word}[defi]{用語}
\newtheorem{ex}[defi]{例}
\newtheorem{lem}[defi]{補題}

\renewcommand\proofname{\bf 証明} % proof環境の設定変更

\begin{document}

% 本文
\section{一様連続}

\begin{screen}
\fbox{1}
$f$を開区間$(0,1)$で定義された函数とする.
$f$が一様連続であるとは,
\begin{quote}
任意の$\epsilon>0$に対して,ある$\delta>0$が存在して,
$|x-y|<\delta$となる任意の$x,y\in(0,1)$に対して,
$|f(x)-f(y)|<\epsilon$が成り立つ
\end{quote}
ことである.このとき,次の問に答えよ.
\begin{itemize}
\item[2.]$f$が一様連続ならば有界であることを示せ.
ここで,$f$が有界であるとは,
\begin{quote}
ある正実数$M>0$が存在して,任意の$x\in(0,1)$に対して,
$|f(x)|<M$が成り立つ
\end{quote}
ことである.
\end{itemize}
\end{screen}

\begin{proof}
$f$は一様連続であるから,ある$\delta>0$が存在して,
$|x-y|<\delta$となる任意の$x,y\in(0,1)$に対して,
$|f(x)-f(y)|<1$が成り立つ.
このとき,
\[ (0,1)\subset\bigcup_{i=1}^NB(x_i,\delta),
ただし B(x_i,\delta):=\{x\in\mathbb{R}:|x_i-x|<\delta\} \]
を満たす$x_1,x_2,\cdots,x_N\in(0,1)$が存在する.
よって,任意の$x\in(0,1)$に対して,ある$i=1,2,\cdots,N$が存在して
$|x_i-x|<\delta$が成り立つから,$M:=\max_{1\leq i\leq N}|f(x_i)|$とおくと,
\[ |f(x)|=|f(x)-f(x_i)+f(x_i)|\leq|f(x)-f(x_i)|+|f(x_i)|<1+|f(x_i)|\leq1+M \]
となる.従って,$f$は有界である.
\end{proof}

なお,証明にあたっては\href{https://math.stackexchange.com/questions/931887/how-to-prove-if-f-is-defined-and-uniformly-continuous-on-a-bounded-set-e-th}{こちら}を参照しました.

\section{広義積分}

\begin{screen}
\fbox{2}
次の問に答えよ.
\begin{itemize}
\item[2.]次の広義積分が収束することを示せ
\[ \int_0^1\frac{1}{[x(1-x)]^{1/3}}\,dx \]
\item[3.]次の広義積分が収束しないことを示せ:
\[ \int_0^\infty\frac{2+\sin{e^x}}{x}\,dx \]
\end{itemize}
\end{screen}

\begin{proof}
\begin{itemize}
\item[2.]広義積分が収束することの定義\footnote{\href{https://lecture.ecc.u-tokyo.ac.jp/~nkiyono/kiyono/12_kata-07.pdf}{こちら}を参照しました.}
を確認する.
\begin{boxnote}\begin{defi}
$a$を実数とし,$b\in\mathbb{R}\cup\{\infty\}$とする.
$[a,b)$に含まれる任意の有界閉区間上で可積分な関数$f$に対して,
広義積分
\[ \int_a^bf(x)\,dx \]
が収束するとは,
\[ \lim_{r\to b-0}\int_a^rf(x)\,dx \]
が存在することをいう.
$(a,b]$で定義された関数についても,同様に
\[ \int_a^bf(x)\,dx=\lim_{r\to a+0}\int_r^bf(x)\,dx \]
と定義する.
また,$(a,b)$で定義された関数については,ある$c\in(a,b)$をとって
\[ \int_a^bf(x)\,dx=\lim_{r\to a+0}\int_r^cf(x)\,dx+\lim_{r\to b-0}\int_c^rf(x)\,dx \]
と定義する.
\end{defi}\end{boxnote}
$0<x\leq1/2$のとき,$1-x\geq1/2$であるから
\[ \frac{1}{[x(1-x)]^{1/3}}=\frac{1}{x^{1/3}(1-x)^{1/3}}
\leq\frac{1}{x^{1/3}(1/2)^{1/3}}=2^{1/3}x^{-1/3} \]
となる.よって,$r_0\in(0,1/2]$に対して
\[ \int_{r_0}^{1/2}\frac{1}{[x(1-x)]^{1/3}}\,dx
\leq2^{1/3}\int_{r_0}^{1/2}x^{-1/3}\,dx
=2^{1/3}\left[\frac{2}{3}x^{2/3}\right]_{r_0}^{1/2}
=\frac{2^{2/3}[1-(2r_0)^{2/3}]}{3} \]
であるから,$r_0\to0+0$としたとき,最左辺の積分は収束する.
また,$1/2\leq x<1$のとき,
\[ \frac{1}{[x(1-x)]^{1/3}}=\frac{1}{x^{1/3}(1-x)^{1/3}}
\leq\frac{1}{(1/2)^{1/3}(1-x)^{1/3}}=2^{1/3}(1-x)^{-1/3} \]
となる.よって,$r_1\in[1/2,1)$に対して
\[ \int_{1/2}^{r_1}\frac{1}{[x(1-x)]^{1/3}}\,dx
\leq2^{1/3}\int_{1/2}^{r_1}(1-x)^{-1/3}\,dx
=2^{1/3}\left[-\frac{2}{3}(1-x)^{2/3}\right]_{1/2}^{r_1}
=\frac{2^{4/3}}{3}[2^{-2/3}-(1-r_1)^{2/3}] \]
であるから,$r_1\to1-0$としたとき,最左辺の積分は収束する.
以上より,広義積分
\[ \int_0^1\frac{1}{[x(1-x)]^{1/3}}\,dx \]
は収束する.

\item[3.]$\epsilon,M>0$とするとき
\[ \int_\epsilon^M\frac{2+\sin{e^x}}{x}\,dx
\geq\int_\epsilon^M\frac{2-1}{x}\,dx
=\int_\epsilon^M\frac{1}{x}\,dx
=\log{M}-\log{\epsilon} \]
であり,
\[ \lim_{M\to\infty}\log{M}=\infty,\lim_{\epsilon\to0+0}\log{\epsilon}=-\infty \]
であるから,広義積分
\[ \int_0^\infty\frac{2+\sin{e^x}}{x}\,dx \]
は収束しない.
\end{itemize}
\end{proof}

なお,2の証明にあたっては友人の解答を参照しました.
感謝申し上げます.

\section{線形部分空間}

\begin{screen}
\fbox{3}
$V$と$W$を$d$次元実線型空間$\mathbb{R}^d$の部分空間とする.
\begin{itemize}
\item[3.]和集合$V\cup W$は$\mathbb{R}^d$の部分空間となるか?
理由とともに答えよ.
\end{itemize}
\end{screen}

$V\cup W$は,
$d=1$のとき$\mathbb{R}^d$の部分空間となり,
$d>1$のとき$\mathbb{R}^d$の部分空間であるとは限らない.

\begin{proof}
$d=1$のとき,$\dim{V},\dim{W}\leq1$であるから,
$V$も$W$も$\{0\}$または$\mathbb{R}$のいずれかである.
よって$V\cup W$も$\{0\}$または$\mathbb{R}$のいずれかである.
従って,$V\cup W$は$\mathbb{R}$の部分空間である.

$d>1$のとき,例えば
$V=\{(x_1,\cdots,x_d)\in\mathbb{R}^d:x_1=0\}$,
$W=\{(x_1,\cdots,x_d)\in\mathbb{R}^d:x_2=0\}$
とすると,$V\cup W$は$\mathbb{R}^d$の部分空間ではない.
実際,$ab\neq0$を満たす$a,b\in\mathbb{R}$に対して,
$(a,0,0,\cdots,0)\in V\subset V\cup W$,$(0,b,0,\cdots,0)\in W\subset V\cup W$
であるが,
$(a,0,0,\cdots,0)+(0,b,0,\cdots,0)=(a,b,0,\cdots,0)\notin V\cup W$
となる.従って,$V\cup W$は$\mathbb{R}^d$の部分空間であるとは限らない.
\end{proof}

上記の証明において用いられている以下の事実も証明しておこう.

\begin{prop}
$V$を有限次元線形空間,$W$を$V$の部分空間とするとき,次が成り立つ.
\begin{enumerate}
\item $\dim{W}\leq\dim{V}$
\item $\dim{W}=\dim{V}$ならば$W=V$
\end{enumerate}
\end{prop}

\begin{proof}
後でやる.
\end{proof}

\section{固有空間}

\begin{screen}
\fbox{4}
二次以下の実数係数多項式の全体からなる実線型空間を$V$として,
$V$上の線型変換$D\colon V\to V$を
\[ f(x)\mapsto\frac{d}{dx}[(1+x)f(x)] \]
と定める.このとき,次の問に答えよ.
\begin{itemize}
\item[3.]$D$の固有空間の全てについて,それぞれの基底を一組ずつ求めよ.
\end{itemize}
\end{screen}

固有空間の定義を再確認しよう.

\begin{boxnote}\begin{defi}
$V$を$K$線形空間,$T: V\to V$を$V$上の線形変換とする.
0でない$v\in V$が$T$の固有ベクトルであるとは,
ある$\lambda\in K$が存在して$T(v)=\lambda v$が成り立つことをいう.
このとき,$\lambda$を固有ベクトル$v$に関する固有値といい,さらに
\[ W_\lambda:=\{w\in V:T(w)=\lambda w\} \]
によって定義される線形空間$W_\lambda$を,固有値$\lambda$に関する固有空間と呼ぶ.
\end{defi}\end{boxnote}

従って,$D$の表現行列を用いて$V$を$\mathbb{R}^3$とみなして考えたときの固有空間は,
$D$の固有空間そのものではないことに注意しよう.
$V$を$\mathbb{R}^3$とみなして考えたとき,固有値1,2,3に関する固有空間の基底の1つは,それぞれ
\[ \begin{pmatrix}1\\0\\0\end{pmatrix},\begin{pmatrix}1\\1\\0\end{pmatrix},
\begin{pmatrix}1\\2\\1\end{pmatrix} \]
であるから,対応する$D$の固有空間の基底は,それぞれ
\[ 1,1+(1+2x),1+2(1+2x)+(2x+3x^2) \]
である.

\end{document}