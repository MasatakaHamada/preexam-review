\documentclass{jsarticle}
\usepackage{amsmath, amssymb} % 数学記号のパッケージ
\usepackage{ascmac} % 囲み枠のパッケージ

% ハイパーリンクに関するパッケージとその設定
\usepackage[dvipdfmx]{hyperref}
\usepackage{pxjahyper}
\hypersetup{
 colorlinks=true
}

% 定理などの番号付けに関するパッケージとその設定
\usepackage{amsthm}
\theoremstyle{definition}
\newtheorem{defi}{定義}[section]
\newtheorem{prop}[defi]{命題}
\newtheorem{thm}[defi]{定理}
\newtheorem{supp}[defi]{補足}
\newtheorem{word}[defi]{用語}
\newtheorem{ex}[defi]{例}
\newtheorem{lem}[defi]{補題}

\renewcommand\proofname{\bf 証明} % proof環境の設定変更

% タイトルなど
\title{2019年度予備テスト 解答と補足}
\author{Masataka HAMADA}

\begin{document}
\maketitle

% 本文
\section{関数の一様連続性}
\subsection{問題と解答}

\begin{screen}
\fbox{1}
$f$を開区間$(0,1)$で定義された函数とする.
$f$が一様連続であるとは,
\begin{quote}
任意の$\epsilon>0$に対して,ある$\delta>0$が存在して,
$|x-y|<\delta$となる任意の$x,y\in(0,1)$に対して,
$|f(x)-f(y)|<\epsilon$が成り立つ
\end{quote}
ことである.このとき,次の問に答えよ.
\begin{enumerate}
\item$f(x)=\sin(1/x)$とするとき,$f$は一様連続か?
理由とともに答えよ.
\item$f$が一様連続ならば有界であることを示せ.
ここで,$f$が有界であるとは,
\begin{quote}
ある正実数$M>0$が存在して,任意の$x\in(0,1)$に対して,
$|f(x)|<M$が成り立つ
\end{quote}
ことである.
\item$f$が一様連続ならば,$x_n\in(0,1)$かつ
$\lim_{n\to\infty}x_n=0$となる任意の数列$\{x_n\}_{n=1}^\infty$に対して,
$\{f(x_n)\}_{n=1}^\infty$がコーシー列になることを示せ.
\end{enumerate}
\end{screen}

\begin{enumerate}
\item$f$は一様連続でない.
実際,$\epsilon=1$とすると,任意の$\delta>0$に対して
\[ \left|\frac{1}{(2n+1/6)(2n+7/6)\pi}\right|<\delta \]
を満たす自然数$n$をとって$x=[(2n+1/6)\pi]^{-1},y=[(2n+7/6)\pi]^{-1}$
とおけば,$x,y\in(0,1)$であり,
\[ |x-y|=\left|\frac{1}{(2n+1/6)(2n+7/6)\pi}\right|<\delta \]
かつ
\[ |f(x)-f(y)|
=\left|\sin\left(2n+\frac{1}{6}\right)\pi-\sin\left(2n+\frac{7}{6}\right)\pi\right|
=\left|\frac{1}{2}+\frac{1}{2}\right|=1\geq\epsilon \]
となる.

\item$f$は一様連続であるから,ある$\delta>0$が存在して,
$|x-y|<\delta$となる任意の$x,y\in(0,1)$に対して,
$|f(x)-f(y)|<1$が成り立つ.
このとき,
\[ (0,1)\subset\bigcup_{i=1}^NB(x_i,\delta),
ただし B(x_i,\delta):=\{x\in\mathbb{R}:|x_i-x|<\delta\} \]
を満たす$x_1,x_2,\cdots,x_N\in(0,1)$が存在する.
よって,任意の$x\in(0,1)$に対して,ある$i=1,2,\cdots,N$が存在して
$|x_i-x|<\delta$が成り立つから,$M:=\max_{1\leq i\leq N}|f(x_i)|$とおくと,
\[ |f(x)|=|f(x)-f(x_i)+f(x_i)|\leq|f(x)-f(x_i)|+|f(x_i)|<1+|f(x_i)|\leq1+M \]
となる.従って,$f$は有界である.

\item 任意の$\epsilon>0$に対して,$f$が一様連続であることから,
ある$\delta>0$が存在して,$|x-y|<\delta$となる任意の$x,y\in(0,1)$に対して
$|f(x)-f(y)|<\epsilon$となる.さらに$\lim_{n\to\infty}x_n=0$なので,
ある自然数$N$が存在して,$n\geq N$となる任意の自然数$n$に対し
$|x_n|<\delta/2$となる.このとき,$m,n\geq N$となる任意の自然数$m,n$について
$|x_m-x_n|\leq|x_m|+|x_n|<\delta/2+\delta/2=\delta$となるから
$|f(x_m)-f(x_n)|<\epsilon$であることが従う.
以上より,$\{f(x_n)\}_{n=1}^\infty$はコーシー列である.
\end{enumerate}

\subsection{補足}

\begin{enumerate}
\item$f$が一様連続でない,すなわち一様連続であることの定義の否定は
\begin{quote}
ある$\epsilon>0$が存在し,任意の$\delta>0$に対して,
$|x-y|<\delta$かつ$|f(x)-f(y)|\geq\epsilon$となる$x,y\in(0,1)$が存在する
\end{quote}
となります.

\item 証明にあたっては\href{https://math.stackexchange.com/questions/931887/how-to-prove-if-f-is-defined-and-uniformly-continuous-on-a-bounded-set-e-th}{こちら}を参照しました.

\item 実数列$\{x_n\}_{n=1}^\infty$がコーシー列であるとは,
\begin{quote}
任意の$\epsilon>0$に対して,ある自然数$N$が存在して,
$m,n\geq N$となる任意の自然数$m,n$に対して
$|x_m-x_n|<\epsilon$が成り立つ
\end{quote}
ことです.
\end{enumerate}

\section{広義積分}
\subsection{問題と解答}

\begin{screen}
\fbox{2}
次の問に答えよ.
\begin{enumerate}
\item 次の広義積分が収束するための実数$p$の充たすべき必要十分条件を求めよ:
\[ \int_0^1(1-x)^p\,dx \]
\item 次の広義積分が収束することを示せ
\[ \int_0^1\frac{1}{[x(1-x)]^{1/3}}\,dx \]
\item 次の広義積分が収束しないことを示せ:
\[ \int_0^\infty\frac{2+\sin{e^x}}{x}\,dx \]
\end{enumerate}
\end{screen}

\begin{enumerate}
\item$p>-1$のとき,
\[ \int_0^1(1-x)^p\,dx=\left[-\frac{(1-x)^{p+1}}{p+1}\right]_0^1=\frac{1}{p+1} \]
となる.

$p=-1$のとき,$0<\epsilon<1$に対して
\[ \int_0^\epsilon(1-x)^p\,dx=\left[-\log(1-x)\right]_0^\epsilon=-\log(1-\epsilon) \]
となるが,$\lim_{x\to0+0}\log{x}=-\infty$であるから
$\lim_{\epsilon\to1-0}[-\log(1-\epsilon)]=\infty$である.つまり,広義積分
\[ \int_0^1(1-x)^p\,dx \]
は発散する.

$p<-1$のとき,$0<\epsilon<1$に対して
\[ \int_0^\epsilon(1-x)^p\,dx=\left[-\frac{(1-x)^{p+1}}{p+1}\right]_0^\epsilon
=-\frac{(1-\epsilon)^{p+1}}{p+1}+\frac{1}{p+1} \]
となるが,$\lim_{x\to0+0}x^{p+1}=\infty$であるから
$\lim_{\epsilon\to1-0}[-\frac{(1-\epsilon)^{p+1}}{p+1}+\frac{1}{p+1}]=-\infty$であるから,
広義積分
\[ \int_0^1(1-x)^p\,dx \]
は発散する.
従って,広義積分
\[ \int_0^1(1-x)^p\,dx \]
が収束するための実数$p$の充たすべき必要十分条件は$p>-1$である.

\item$0<x\leq1/2$のとき,$1-x\geq1/2$であるから
\[ \frac{1}{[x(1-x)]^{1/3}}=\frac{1}{x^{1/3}(1-x)^{1/3}}
\leq\frac{1}{x^{1/3}(1/2)^{1/3}}=2^{1/3}x^{-1/3} \]
となる.よって,$r_0\in(0,1/2]$に対して
\[ \int_{r_0}^{1/2}\frac{1}{[x(1-x)]^{1/3}}\,dx
\leq2^{1/3}\int_{r_0}^{1/2}x^{-1/3}\,dx
=2^{1/3}\left[\frac{2}{3}x^{2/3}\right]_{r_0}^{1/2}
=\frac{2^{2/3}[1-(2r_0)^{2/3}]}{3} \]
であるから,$r_0\to0+0$としたとき,最左辺の積分は収束する.
また,$1/2\leq x<1$のとき,
\[ \frac{1}{[x(1-x)]^{1/3}}=\frac{1}{x^{1/3}(1-x)^{1/3}}
\leq\frac{1}{(1/2)^{1/3}(1-x)^{1/3}}=2^{1/3}(1-x)^{-1/3} \]
となる.よって,$r_1\in[1/2,1)$に対して
\[ \int_{1/2}^{r_1}\frac{1}{[x(1-x)]^{1/3}}\,dx
\leq2^{1/3}\int_{1/2}^{r_1}(1-x)^{-1/3}\,dx
=2^{1/3}\left[-\frac{2}{3}(1-x)^{2/3}\right]_{1/2}^{r_1}
=\frac{2^{4/3}}{3}[2^{-2/3}-(1-r_1)^{2/3}] \]
であるから,$r_1\to1-0$としたとき,最左辺の積分は収束する.
以上より,広義積分
\[ \int_0^1\frac{1}{[x(1-x)]^{1/3}}\,dx \]
は収束する.

\item$\epsilon,M>0$とするとき
\[ \int_\epsilon^M\frac{2+\sin{e^x}}{x}\,dx
\geq\int_\epsilon^M\frac{2-1}{x}\,dx
=\int_\epsilon^M\frac{1}{x}\,dx
=\log{M}-\log{\epsilon} \]
であり,
\[ \lim_{M\to\infty}\log{M}=\infty,\lim_{\epsilon\to0+0}\log{\epsilon}=-\infty \]
であるから,広義積分
\[ \int_0^\infty\frac{2+\sin{e^x}}{x}\,dx \]
は収束しない.
\end{enumerate}

\subsection{補足}

広義積分が収束することの定義は次の通りです.
\begin{defi}
$a$を実数とし,$b\in\mathbb{R}\cup\{\infty\}$とする.
$[a,b)$に含まれる任意の有界閉区間上で可積分な関数$f$に対して,
広義積分
\[ \int_a^bf(x)\,dx \]
が収束するとは,
\[ \lim_{r\to b-0}\int_a^rf(x)\,dx \]
が存在することをいう.
$(a,b]$で定義された関数についても,同様に
\[ \int_a^bf(x)\,dx=\lim_{r\to a+0}\int_r^bf(x)\,dx \]
と定義する.
また,$(a,b)$で定義された関数については,ある$c\in(a,b)$をとって
\[ \int_a^bf(x)\,dx=\lim_{r\to a+0}\int_r^cf(x)\,dx+\lim_{r\to b-0}\int_c^rf(x)\,dx \]
と定義する.
\end{defi}

なお,2の証明にあたっては友人の解答を参照しました.
感謝申し上げます.

\section{線形部分空間}
\subsection{問題と解答}

\begin{screen}
\fbox{3}
$V$と$W$を$d$次元実線型空間$\mathbb{R}^d$の部分空間とする.
和$V+W$と交わり$V\cap W$も$\mathbb{R}^d$の部分空間となる.
$\mathbb{R}^d$の元$u_1,\cdots,u_\ell,v_1,\cdots,v_m,w_1,\cdots,w_n$を次のように選ぶ:
\begin{itemize}
\item$u_1,\cdots,u_\ell$は$V\cap W$の基底である.
\item$u_1,\cdots,u_\ell,v_1,\cdots,v_m$は$V$の基底である.
\item$u_1,\cdots,u_\ell,w_1,\cdots,w_n$は$W$の基底である.
\end{itemize}
このとき,次の問に答えよ.
\begin{enumerate}
\item 和$V+W$の任意の元が$u_1,\cdots,u_\ell,v_1,\cdots,v_m,w_1,\cdots,w_n$の
線型結合で表されることを示せ.
\item 元$u_1,\cdots,u_\ell,v_1,\cdots,v_m,w_1,\cdots,w_n$は線型独立であることを示せ.
\item 和集合$V\cup W$は$\mathbb{R}^d$の部分空間となるか?
理由とともに答えよ.
\end{enumerate}
\end{screen}

\begin{enumerate}
\item 任意の$v\in V,w\in W$は,
ある$a_1,\cdots,a_{\ell+m},b_1,\cdots,b_{\ell+n}\in\mathbb{R}$を用いて
\[ v=\sum_{i=1}^\ell a_iu_i+\sum_{j=1}^ma_{\ell+j}v_j,
w=\sum_{i=1}^\ell b_iu_i+\sum_{k=1}^nb_{\ell+k}w_k \]
と表すことができ,このとき
\[ v+w=\sum_{i=1}^\ell(a_i+b_i)u_i+\sum_{j=1}^ma_{\ell+j}v_j+\sum_{k=1}^nb_{\ell+k}w_k \]
となる.
従って,$V+W$の任意の元は$u_1,\cdots,u_\ell,v_1,\cdots,v_m,w_1,\cdots,w_n$の
線型結合で表される.

\item$a_1,\cdots,a_\ell,b_1,\cdots,b_m,c_1,\cdots,c_n\in\mathbb{R}$に対し
\begin{equation}\label{3-1}
\sum_{i=1}^\ell a_iu_i+\sum_{j=1}^mb_jv_j+\sum_{k=1}^nc_kw_k=0
\end{equation}
であるとする.このとき
\[ v=\sum_{i=1}^\ell a_iu_i+\sum_{j=1}^mb_jv_j \]
とおくと,$v\in V$である.さらに,(\ref{3-1})より
\[ v=-\sum_{k=1}^nc_kw_k=\sum_{k=1}^n(-c_k)w_k \]
であるから$v\in W$である.よって$v\in V\cap W$であるから,
$v=d_1u_1+\cdots+d_\ell u_\ell$となる$d_1,\cdots,d_\ell\in\mathbb{R}$が存在する.つまり
\[ \sum_{i=1}^\ell a_iu_i+\sum_{j=1}^mb_jv_j=\sum_{i=1}^\ell d_iu_i \]
であるから
\begin{equation}\label{3-2}
\sum_{i=1}^\ell(a_i-d_i)u_i+\sum_{j=1}^mb_jv_j=0
\end{equation}
となる.ここで,$u_1,\cdots,u_\ell,v_1,\cdots,v_m$は線型独立であるから,
(\ref{3-2})より特に$b_1=\cdots=b_m=0$である.よって(\ref{3-1})は
\[ \sum_{i=1}^\ell a_iu_i+\sum_{k=1}^nc_kw_k=0 \]
となるが,$u_1,\cdots,u_\ell,w_1,\cdots,w_n$も線型独立であるから,これより
$a_1=\cdots=a_\ell=c_1=\cdots=c_n=0$である.

以上より,$u_1,\cdots,u_\ell,v_1,\cdots,v_m,w_1,\cdots,w_n$は線型独立である.

\item$d=1$のとき,$\dim{V},\dim{W}\leq1$であるから,
$V$も$W$も$\{0\}$または$\mathbb{R}$のいずれかである.
よって$V\cup W$も$\{0\}$または$\mathbb{R}$のいずれかである.
従って,$V\cup W$は$\mathbb{R}$の部分空間である.

$d>1$のとき,例えば
$V=\{(x_1,\cdots,x_d)\in\mathbb{R}^d:x_1=0\}$,
$W=\{(x_1,\cdots,x_d)\in\mathbb{R}^d:x_2=0\}$
とすると,$V\cup W$は$\mathbb{R}^d$の部分空間ではない.
実際,$ab\neq0$を満たす$a,b\in\mathbb{R}$に対して,
$(a,0,0,\cdots,0)\in V\subset V\cup W$,$(0,b,0,\cdots,0)\in W\subset V\cup W$
であるが,
$(a,0,0,\cdots,0)+(0,b,0,\cdots,0)=(a,b,0,\cdots,0)\notin V\cup W$
となる.従って,$V\cup W$は$\mathbb{R}^d$の部分空間であるとは限らない.

以上より,$V\cup W$は,
$d=1$のとき$\mathbb{R}^d$の部分空間となり,
$d>1$のとき$\mathbb{R}^d$の部分空間であるとは限らない.
\end{enumerate}

\subsection{補足}

3の解答では次の命題の1を用いています.

\begin{prop}\label{dim}
$V$を有限次元線形空間,$W$を$V$の部分空間とするとき,次が成り立つ.
\begin{enumerate}
\item $W$も有限次元であり,$\dim{W}\leq\dim{V}$
\item $\dim{W}=\dim{V}$ならば$W=V$
\end{enumerate}
\end{prop}

証明は,例えば
\href{http://www.utp.or.jp/book/b305671.html}{「線形代数の世界」(斎藤毅,東京大学出版会,2007)}の1.5節
を参照してください.

\section{核と像,表現行列,固有空間}
\subsection{問題と解答}

\begin{screen}
\fbox{4}
二次以下の実数係数多項式の全体からなる実線型空間を$V$として,
$V$上の線型変換$D\colon V\to V$を
\[ f(x)\mapsto\frac{d}{dx}[(1+x)f(x)] \]
と定める.このとき,次の問に答えよ.
\begin{enumerate}
\item$\mathrm{Ker}\,D$と$\mathrm{Im}\,D$を求めよ.
\item$V$の基底$\{1,1+2x,2x+3x^2\}$に関する$D$の表現行列を求めよ.
\item$D$の固有空間の全てについて,それぞれの基底を一組ずつ求めよ.
\end{enumerate}
\end{screen}

\begin{enumerate}
\item$a,b,c\in\mathbb{R}$とし,$f(x)=ax^2+bx+c\in V$とすると
\[ (Df)(x)=\frac{d}{dx}[(1+x)(ax^2+bx+c)]=3ax^2+2(a+b)x+b+c \]
となる.ここで
\[ \begin{cases}3a&=0 \\ 2(a+b)&=0 \\ b+c&=0\end{cases} \]
の解は$a=b=c=0$であるから,$\mathrm{Ker}\,D=\{0\}$である.

また,$\mathrm{Im}\,D=V$である.実際,任意の$f\in V$が
$p,q,r\in\mathbb{R}$を用いて$f(x)=px^2+qx+r$と表されるとき,
\[ a=\frac{p}{3},b=\frac{q}{2}-\frac{p}{3},c=r-\frac{q}{2}+\frac{p}{3} \]
とおけば
\[ f(x)=3ax^2+2(a+b)x+b+c=(D(ax^2+bx+c))(x) \]
となるから$f\in\mathrm{Im}\,D$である.

\item1での計算結果を用いて
\begin{itemize}
\item$(D1)(x)=1=1\cdot1+0\cdot(1+2x)+0\cdot(2x+3x^2)$
\item$(D(1+2x))(x)=4x+3=1\cdot1+2\cdot(1+2x)+0\cdot(2x+3x^2)$
\item$(D(2x+3x^2))(x)=9x^2+10x+2=0\cdot1+2\cdot(1+2x)+3\cdot(2x+3x^2)$
\end{itemize}
であることがわかる.
従って,$V$の基底$\{1,1+2x,2x+3x^2\}$に関する$D$の表現行列は
\[ \begin{pmatrix}1&1&0 \\ 0&2&2 \\ 0&0&3\end{pmatrix} \]
である.

\item2の表現行列を$A$とする.$A$の固有多項式$\phi(x)$は
\[ \phi(x)
=\begin{vmatrix}x-1&-1&0 \\ 0&x-2&-2 \\ 0&0&x-3\end{vmatrix}
=(x-1)\begin{vmatrix}x-2&-2 \\ 0&x-3\end{vmatrix}
=(x-1)(x-2)(x-3) \]
であるから,$A$の固有値は1,2,3である.
以下,$x=\begin{pmatrix}x_1 \\ x_2 \\ x_3\end{pmatrix}\in\mathbb{R}^3$とする.

\[ Ax=x\Leftrightarrow
\begin{cases}x_1+x_2&=x_1 \\ 2x_2+2x_3&=x_2 \\ 3x_3&=x_3\end{cases}
\Leftrightarrow\begin{cases}x_1\in\mathbb{R} \\ x_2=x_3=0\end{cases} \]
であるから,固有値1に関する固有空間の基底の1つは1である.

\[ Ax=2x\Leftrightarrow
\begin{cases}x_1+x_2&=2x_1 \\ 2x_2+2x_3&=2x_2 \\ 3x_3&=2x_3\end{cases}
\Leftrightarrow\begin{cases}x_1=x_2 \\ x_3=0\end{cases} \]
であるから,固有値2に関する固有空間の基底の1つは$1+(1+2x)$である.

\[ Ax=3x\Leftrightarrow
\begin{cases}x_1+x_2&=3x_1 \\ 2x_2+2x_3&=3x_2 \\ 3x_3&=3x_3\end{cases}
\Leftrightarrow\begin{cases}x_1=x_3 \\ x_2=2x_1\end{cases} \]
であるから,固有値3に関する固有空間の基底の1つは$1+2(1+2x)+(2x+3x^2)$である.
\end{enumerate}

\subsection{補足}

\begin{enumerate}
\item(別解)命題\ref{dim}と次の命題を用いると,$\mathrm{Ker}\,D=\{0\}$であることから
$\mathrm{Im}\,D=V$が直ちに従います.

\begin{prop}
有限次元線形空間$V,W$および線形変換$f\colon V\to W$に対し,
次の等式が成り立つ:
\[ \dim{V}=\dim\mathrm{Ker}\,f+\mathrm{rank}\,f \]
\end{prop}

証明は,例えば
\href{http://www.utp.or.jp/book/b302039.html}{「線型代数入門」(齋藤正彦,東京大学出版会,1966)}の第4章\S4[4.5]
や
\href{http://www.utp.or.jp/book/b305671.html}{「線形代数の世界」(斎藤毅,東京大学出版会,2007)}の2.4節
を参照してください.

\item 特にありません.

\item 固有空間の定義は次の通りです.

\begin{defi}
$V$を$K$線形空間,$T: V\to V$を$V$上の線形変換とする.
0でない$v\in V$が$T$の固有ベクトルであるとは,
ある$\lambda\in K$が存在して$T(v)=\lambda v$が成り立つことをいう.
このとき,$\lambda$を固有ベクトル$v$に関する固有値といい,さらに
\[ W_\lambda:=\{w\in V:T(w)=\lambda w\} \]
によって定義される線形空間$W_\lambda$を,固有値$\lambda$に関する固有空間と呼ぶ.
\end{defi}

従って,$D$の表現行列を用いて$V$を$\mathbb{R}^3$とみなして考えたときの固有空間は,
$D$の固有空間そのものではありません.
$V$を$\mathbb{R}^3$とみなして考えたとき,固有値1,2,3に関する固有空間の基底の1つは,それぞれ
\[ \begin{pmatrix}1\\0\\0\end{pmatrix},\begin{pmatrix}1\\1\\0\end{pmatrix},
\begin{pmatrix}1\\2\\1\end{pmatrix} \]
なので,対応する$D$の固有空間の基底は,それぞれ
\[ 1,1+(1+2x),1+2(1+2x)+(2x+3x^2) \]
となります.
\end{enumerate}

\end{document}